% $Id: introduction.tex 34630 2013-04-29 22:53:51Z roldeman $

\section{Beyond the Standard Model}
\label{sec:Introduction}

The Standard Model (SM) is the most successful scientific theory to date \cite{Salam1964}\cite{Glashow1961}\cite{Weinberg1967}. It makes accurate predictions about the physical world that have consistently stood up to experimental scrutiny. This culminated in the discovery of a $125$~GeV particle consistent with a Higgs boson at the Large Hadron Collider (LHC) in 2012~\cite{ATLASHiggs2012}\cite{CMS2012HiggsPaper}. Despite this, the SM is incomplete; neither including a description of gravity or dark energy, nor providing a suitable dark matter candidate.
\\\\
If there is to be a theory of everything, it is assumed new physics must be present at the Planck scale\footnote{Around $1.22\times 10^{19}$GeV}, where gravity becomes relevant. The mass of the Higgs receives quantum corrections from the virtual effects of every particle that couples to it. If new physics exists at the Planck scale it will have an overwhelmingly large contribution to the mass of the Higgs. As the Higgs exists at the electroweak scale, the quantum corrections must mostly cancel each other out. The only way to achieve this without new physics at the electroweak scale is with an incredible fine tuning of parameters, which is known as the ``hierarchy problem'' \cite{SUSYprimerMartin:1997ns}. 
\\\\
A natural cancellation of corrections to the Higgs mass can be provided by the introduction of a new broken spacetime symmetry between fermions and bosons known as Supersymmetry (SUSY). As fermions and bosons provide opposite sign corrections to the Higgs mass, the existence of SUSY close the electroweak scale solves the hierarchy problem. Additionally, in the case that R-parity\footnote{$R\equiv (-1)^{3(B-L)+2s}$, where $B$ is the baryon number, $L$ the lepton number and $s$ the spin.} is conserved, SUSY models can provide a candidate for Dark Matter in the form of a weakly interacting lightest supersymmetric particle (LSP). Another strong theoretical motivation for SUSY is the fact that it may also unify the strong, weak and electromagnetic forces at the GUT scale\footnote{The energy above which the strong, weak and electromagnetic forces are unified.}; this is not possible in the SM. The supersymmetric extension of the SM that introduces the fewest new particles exhibits the above features and is known as the Minimal Supersymmetric Standard Model (MSSM) \cite{SUSYprimerMartin:1997ns}.
