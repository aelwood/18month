% $Id: introduction.tex 34630 2013-04-29 22:53:51Z roldeman $

\section{The $\alpha_T$ analysis framework}
\label{sec:analysisFw}

To carry out an analysis on the data collected by the CMS detector the particles emerging from each collision must be reconstructed with software. Each of the collisions are separated into ``events'' based on timing information from the LHC and the pulse shapes of energy deposits within the detector. Algorithms take the inputs from each of the subdetectors per event and use them to reconstruct the different particles produced. A muon, for example, is characterised by a coincidental track and hit in the muon chambers. As there are $O(10^6)$ events in a typical dataset, this reconstruction must be carried out reasonably quickly using custom built software. The CMS collaboration centralises the reconstruction with in the ``CMS Software'' (CMSSW). 
\\\\
Within the collaboration subgroups of researchers work to define the individual reconstruction algorithms. This process results in a fixed set of algorithms to be used in the different physics analyses. Carrying out the reconstruction can be computationally expensive, so the predefined reconstruction is carried out centrally with events grouped into datasets based on their content. In Run~1, a base level of reconstruction was carried out on raw data from CMS to form the ``AOD'' data tier. Analysis groups then carried out their own further reconstruction on the AOD data. Despite the slimming down from the raw data, the AOD tier still has a lot of information unused by many analyses and incomplete reconstruction on many objects to allow for further algorithm customisation. In Run~2 it is proposed to introduce a further data tier ``miniAOD'' that removes the unnecessary information from the AOD and carries out the further level of reconstruction. This results in data sets that do not require much further processing by the physics analysis groups. The net result is a reduction in the total use of computing time and data storage used by the whole collaboration.
\\\\
To be able to interpret data in the miniAOD format, the $\alpha_T$ analysis framework has been changed significantly. The extra reconstruction and selection required by the analysis have been ported from the old ``ICF'' framework to the ``CMG tools'' framework, initially developed by the CERN CMS subgroup. The outputs of CMG tools are very small flat ROOT trees with the base selections in place and only the specific information required by our analysis.
\\\\
With the flat trees produced, it is possible to carry out the high level analysis work flows. These were also ported to the new ``AlphaTools'' analysis framework and include the systematic error determination through closure tests, accurate estimation of b jet yields in each dataset with the ``b-tag formula method'' \cite{btagformula}, and the statistical interpretation of the results to set limits on or discover particular SUSY processes.