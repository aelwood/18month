\section{Further Work}
\label{sec:conclusion}

SUSY remains a good candidate for physics beyond the SM. The upgrade of the LHC in 2015 therefore puts CMS in a prime position to discover SUSY if it exists at the electroweak scale. The $\alpha_T$ analysis, investigates an important production channel. Work has started on the triggers that will be used during 2015, with studies into a cut on the ratio $\frac{\cancel{H}_T}{H_T}$ as a L1 trigger. It will be necessary to find a reasonable threshold on this new variable, and propagate it through to HLT where other new triggers can be studied. With this complete, we will prepare the analysis framework for the analysis of data taken in Run 2.
\\\\
Interpreting the data taken at the LHC in terms of SUSY is a non trivial task. A framework is being developed that takes the data from all non overlapping CMS SUSY analyses and allows one to study the exclusion of a particular simplified model from the combination of all these analyses. As these analyses are interpreted in terms of simplified models they are being validated within the framework by reproducing the results presented in a paper with simulated data.
\\\\
A study of the proposed upgraded jet algorithm and PUS techniques is well under way. The results so far present a large improvement on the algorithm that is already available. Global $\rho$ and a seed plus donut subtraction are both promising algorithms with different advantages and disadvantages. Further studies will be required to decide on the final PUS technique to be implemented, but a good baseline has been established.