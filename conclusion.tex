\section{Conclusions and further work}
\label{sec:conclusion}

The $\alpha_T$ analysis investigates an important production channel of electroweak scale SUSY. The preparations of the $\alpha_T$ analysis framework for Run~2 of the LHC is nearing completion. Several analysis optimisations have been carried out, improving the sensitivity of the analysis on benchmark SUSY models. Along with this, the effect of new physics object reconstruction algorithms on the analysis have been investigated. In general a moderate improvement is observed.
\\\\
Further work is required to fully understand the implications of the categorisation of events by $\cancel{H_T}$. The change of the systematic error in this dimension, along with the statistical power of the control samples should be fully determined. The aim always being to ensure the $\alpha_T$ analysis is robust and data driven.
\\\\
With the advent of Run~2 data, the analysis methods and frameworks will be tested and validated. Once enough data have been collected, it will be possible to carry out the full analysis. From the preliminary look with $13$~TeV simulated data, it should be possible to start setting limits on the production of new SUSY processes early into Run~2. This gives a potential for discovery around 2016. 