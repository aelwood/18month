% $Id: introduction.tex 34630 2013-04-29 22:53:51Z roldeman $

\section{Sensitivity to SUSY models with \boldmath $13$~TeV Monte Carlo}
\label{sec:phys14Results}

The projected sensitivity to a variety of benchmark SUSY models was calculated with $13$~TeV MC, testing the full analysis work flow. The models were chosen to have SUSY masses at the edge of the masses excluded in Run~1 with $20/$fb of data. A binned likelihood method was utilised to calculate the $95\%$ Confidence Level (CL) limits on the signal strength. This was carried out for scenarios with $1/$fb and $10/$fb of data. The results are summarised in Table~\ref{tab:results}. In this case, a limit of $1.0$ is equivalent to excluding this model to the $95\%$ CL, greater than $1.0$ means it will not be excluded. With only $1/$fb of data the limits are not very strong, not improving much on the results from Run~1 data \cite{susyRun1Twiki}. However, with $10/$fb, collectable by the end of 2015, there is a good degree of exclusion for most models.
 
\begin{table}[h]
\caption{The expected $95\%$ Confidence Level limits on the signal strength for a variety of benchmark SUSY models. \label{tab:results}}
\centering 
\begin{tabular}{ | l | l | l | }
\hline
	&lumi.=1/fb & lumi.=10/fb  \  \\ \hline
	Model & 95\% CL & 95\% CL \\ \hline
	T2tt $m_{\tilde{t}}=650$ $m_{LSP}=325$ & 3.8 & 1.1 \\ 
	T2tt $m_{\tilde{t}}=500$ $m_{LSP}=325$ & 3.2 & 0.9 \\ 
	T2tt $m_{\tilde{t}}=425$ $m_{LSP}=325$ & 2.3 & 0.7 \\ 
	T2qq $m_{\tilde{t}}=600$ $m_{LSP}=550$ & 0.6 & 0.2 \\ 
	T2qq $m_{\tilde{t}}=1200$ $m_{LSP}=100$ & 2.9 & 0.8 \\ 
	T2bb $m_{\tilde{t}}=900$ $m_{LSP}=100$ & 5.4 & 1.2 \\ 
	T2bb $m_{\tilde{t}}=600$ $m_{LSP}=580$ & 9.3 & 2.6 \\ 
	T1tttt $m_{\tilde{g}}=1500$ $m_{LSP}=100$ & 3.8 & 0.6 \\ 
	T1tttt $m_{\tilde{g}}=1200$ $m_{LSP}=800$ & 4.6 & 1.1 \\ 
	T1qqqq $m_{\tilde{g}}=1400$ $m_{LSP}=800$ & 1.5 & 0.4 \\ 
	T1qqqq $m_{\tilde{g}}=1000$ $m_{LSP}=800$ & 1.0 & 0.3 \\ 
	T1bbbb $m_{\tilde{g}}=1500$ $m_{LSP}=100$ & 1.5 & 0.3 \\ 
	T1bbbb $m_{\tilde{g}}=1000$ $m_{LSP}=900$ & 1.2 & 0.3 \\ \hline
\end{tabular}
\end{table}